\section*{H1/6. feladat}
\addcontentsline{toc}{section}{H1/6. feladat}

Határozza meg azt a fordulatszámot \SI{}{\1\per\minute}-ban, amelynél az $A$ szár éppen kiürül, ha nyugalmi helyzetben a víz $H = \SI{0,4}{\meter}$ magasan áll az U csőben, $L = \SI{0,4}{\meter}$, $\varrho = \SI{e3}{\kilogram\per\meter\cubed}$, $g = \SI{9,81}{\meter\per\second\squared}$, valamint ennél a fordulatszámnál a $C$ pontban a nyomást $\SI{}{\pascal}$ -ban, ha $p_0 = \SI{1}{\bar}$! Készítsen vázlatot! A víz nem lép ki az U-csőből.

\subsubsection* {Megoldási útmutatás:}
\vspace{2mm}
Rajzolja be a keresett fordulatszámnál a vízfelszínt és a rajta keresztülmenő potenciálfelszínt!
\\ \\
Válasszon alkalmas helyen az $r$ koordinátát! Írja fel a potenciálfelszín metszetét leíró $h= f(r)$ függvényt!

\subsection{Ábrák:}

	\begin{tikzpicture}

\pgfmathsetmacro{\L}{4}
\pgfmathsetmacro{\H}{3}

\pgfmathsetmacro{\d}{0.7} % diameter of the tube
\pgfmathsetmacro{\r}{1} % elbow radius
\pgfmathsetmacro{\h}{5} % height of tube
\pgfmathsetmacro{\p}{0.7} % pivot size

% Pipe
\draw [very thick]   
(0, \d/2)
-- ({\L - \d/2 - (\r-\d/2)}, \d/2)
arc (270:360:\r-\d/2)
-- ( \L - \d/2, \h)
( \L + \d/2, \h)
-- ( \L + \d/2, {-\d/2 + (\r+\d/2)})
arc (0:-90:\r+\d/2)
-- (-\L - \d/2 + \r+\d/2, -\d/2)
arc (270:180:\r+\d/2)
-- (-\L - \d/2, \h)
(-\L + \d/2, \h)
-- (-\L + \d/2, \d/2+\r-\d/2)
arc (180:270:\r-\d/2)
-- (0, \d/2);

% Fluid
\fill [pattern=horizontal lines]
(0, \d/2)
-- ({\L - \d/2 - (\r-\d/2)}, \d/2)
arc (270:360:\r-\d/2)
-- ( \L - \d/2, \H)
-- ( \L + \d/2, \H)
-- ( \L + \d/2, {-\d/2 + (\r+\d/2)})
arc (0:-90:\r+\d/2)
-- (-\L - \d/2 + \r+\d/2, -\d/2)
arc (270:180:\r+\d/2)
-- (-\L - \d/2, \H)
-- (-\L + \d/2, \H)
-- (-\L + \d/2, \d/2+\r-\d/2)
arc (180:270:\r-\d/2)
-- cycle;

% Centerline
\draw [dashdotted, thick]
(-\L, \h*1.1)
-- (-\L, \r)
arc (180:270:\r)
-- (\L - \r, 0)
arc (-90:0:\r)
-- (\L, \h*1.1);

% Center axis
\draw [dashdotted] (0, -0.5*\h) -- (0, 1.5*\h);
% Pivot axis
\draw [dashdotted] (-\L/2, -0.5*\h) -- (-\L/2, 1.5*\h);

% L/2 dimension
\draw [arrows={triangle 45-triangle 45}]
(-\L/2, -0.7*\H) -- (0, -0.7*\H); 
\draw (-\L/4, -0.7*\H) node [anchor=south] {\LARGE L/2};
% L dimensions
\draw [arrows={triangle 45-triangle 45}]
(0, 0.7*\h) --++ (-\L, 0); 
\draw (-\L/2, 0.7*\h) node [anchor=south east] {\LARGE L};
\draw [arrows={triangle 45-triangle 45}]
(0, 0.7*\h) --++ (\L, 0); 
\draw (\L/2, 0.7*\h) node [anchor=south west] {\LARGE L};
% H dimension
\draw (\L - \r, 0) --++ (\r + 0.5*\H + 0.5, 0);
\draw (-\L, \H) --++ (2*\L + 0.5*\H + 0.5, 0);
\draw [arrows={triangle 45-triangle 45}]
(\L + 0.5*\H, 0) --++ (0, \H);
\draw (\L + 0.5*\H, \H/2) node [anchor=west] {\LARGE H};
% rho-V
\draw (\L/4, 0) --++ (1,1);
\draw (\L/4 + 1, 1) node [anchor=south west]
{\LARGE $\rho_V$};

% Pivot
\draw [very thick]
(-\L/2 - 0.1, -\d/2 - 0.1 - \p)
-- (-\L/2 - 0.1, -\d/2 - 0.1 - 0.3*\p)
arc (0:90:0.3*\p)
-- (-\L/2 - 0.1 - \p, -\d/2 - 0.1);
\draw [very thick]
(-\L/2 + 0.1, -\d/2 - 0.1 - \p)
-- (-\L/2 + 0.1, -\d/2 - 0.1 - 0.3*\p)
arc (180:90:0.3*\p)
-- (-\L/2 + 0.1 + \p, -\d/2 - 0.1);	

% C
\draw (\L - \r, \r) ++ (0.7*\r,-0.7*\r) --++ (1,-1);
\draw (\L - \r, \r) ++ (0.7*\r,-0.7*\r)
[fill=black] circle (0.1);
\draw (\L - \r, \r) ++ (0.7*\r,-0.7*\r) ++ (1,-1)
[fill=white] circle (0.3);
\draw (\L - \r, \r) ++ (0.7*\r,-0.7*\r) ++ (1,-1) node {C};

% A
\draw (-\L-\d/2, 0.8*\h) --++ (-1,1);
\draw (-\L-\d/2, 0.8*\h) ++ (-1,1) [fill=white] circle (0.3);
\draw (-\L-\d/2, 0.8*\h) ++ (-1,1) node {A};
% B
\draw (\L+\d/2, 0.8*\h) --++ (1,1);
\draw (\L+\d/2, 0.8*\h) ++ (1,1) [fill=white] circle (0.3);
\draw (\L+\d/2, 0.8*\h) ++ (1,1) node {B};

% P0's
\draw (-\L, 1.1*\h) node [anchor=south] {\large $P_0$};
\draw (\L, 1.1*\h) node [anchor=south] {\large $P_0$};


\end{tikzpicture}

\begin{tikzpicture}

\pgfmathsetmacro{\L}{3}
\pgfmathsetmacro{\H}{5}

\pgfmathsetmacro{\d}{0.7} % diameter of the tube
\pgfmathsetmacro{\r}{1} % elbow radius
\pgfmathsetmacro{\h}{5} % height of tube
\pgfmathsetmacro{\p}{0.7} % pivot size

% Pipe
\draw [very thick]   
(0, \d/2)
-- ({\L - \d/2 - (\r-\d/2)}, \d/2)
arc (270:360:\r-\d/2)
-- ( \L - \d/2, \h)
-- ( \L + \d/2, \h)
-- ( \L + \d/2, {-\d/2 + (\r+\d/2)})
arc (0:-90:\r+\d/2)
-- (-\L - \d/2 + \r+\d/2, -\d/2)
arc (270:180:\r+\d/2)
-- (-\L - \d/2, \h)
-- (-\L + \d/2, \h)
-- (-\L + \d/2, \d/2+\r-\d/2)
arc (180:270:\r-\d/2)
-- cycle;

% Centerline
\draw [dashdotted, thick]
(-\L, \h*1.1)
-- (-\L, \r)
arc (180:270:\r)
-- (\L - \r, 0)
arc (-90:0:\r)
-- (\L, \h*1.1);

% Pivot axis
\draw [dashdotted, arrows={-triangle 45}]
(-\L/2, -0.5*\h) -- (-\L/2, 1.5*\h);
\draw (-\L/2, 1.5*\h) node [right] {h};

% Pivot
\draw [very thick]
(-\L/2 - 0.1, -\d/2 - 0.1 - \p)
-- (-\L/2 - 0.1, -\d/2 - 0.1 - 0.3*\p)
arc (0:90:0.3*\p)
-- (-\L/2 - 0.1 - \p, -\d/2 - 0.1);
\draw [very thick]
(-\L/2 + 0.1, -\d/2 - 0.1 - \p)
-- (-\L/2 + 0.1, -\d/2 - 0.1 - 0.3*\p)
arc (180:90:0.3*\p)
-- (-\L/2 + 0.1 + \p, -\d/2 - 0.1);	
\draw [thick, arrows={-triangle 45}]
([shift=(180:0.6)]-\L/2+0.45,-\d/2-1.5*\p)
arc (100:440:0.6 and 0.2);

% C
\draw (\L - \r, \r) ++ (0.7*\r,-0.7*\r) --++ (1,-1);
\draw (\L - \r, \r) ++ (0.7*\r,-0.7*\r)
[fill=black] circle (0.1);
\draw (\L - \r, \r) ++ (0.7*\r,-0.7*\r) ++ (1,-1)
[fill=white] circle (0.3);
\draw (\L - \r, \r) ++ (0.7*\r,-0.7*\r) ++ (1,-1) node {C};


% Parabola
\draw (-4,1) parabola bend (-1.5,-.5) (4,4);

% 2H
\draw (\L+\d/4, 0.4*\h) ++ (0,4*\d/4,+0,4*\d/4) --++ (1,-1);
\draw (\L+\d/4, 0.4*\h) ++ (0,4*\d/4,+0,4*\d/4)
[fill=black] circle (0.1);
\draw (\L+\d/4, 0.4*\h) ++ (0,4*\d/4,+0,4*\d/4) ++ (1,-1)
[fill=white] circle (0.3);
\draw (\L+\d/4, 0.4*\h) ++ (0,4*\d/4,+0,4*\d/4) ++ (1,-1) node {2H};

% 0
\draw (-\L - \r, \r) ++ (\r,-\r) --++ (1,1);
\draw (-\L - \r, \r) ++ (\r,-\r) ++ (1,1)
[fill=white] circle (0.3);
\draw (-\L - \r, \r) ++ (\r,-\r) ++ (1,1) node {0};

\end{tikzpicture}

\vspace{5mm}

A fenti ábrán látható a víz potenciálgörbéje, így a forgás hatására az $A$ szár kiürül, ezt vehetjük a $0$ szintnek; valamint ennek hatására a $B$ szárban az eredeti vízszint kétszeresére nő azaz $2H$-ra.

\subsubsection* {Adatok:}

\begin{equation}
L = \SI{0,4}{\meter}
\quad
p_0 = \SI{1}{bar},
\quad
\varrho_v = \SI{e3}{\kilogram\per\meter\cubed}
\end{equation}
\begin{equation}
g = \SI{9,81}{\meter\per\second\squared},
\quad
H = \SI{0,4}{\meter},
\end{equation}

\vspace{20mm}
A térerővektor alakja:

\[
\vec{F_t}=
\left[ {\begin{array}{ccc}
	\omega^2 r \\
	0 \\
	-g \\
	\end{array} } \right]
\]
	
\subsubsection* {a) Fordulatszám meghatározása \SI{}{\1\per\minute}-ban:}


\begin{equation}
p_0 - p_2H = \int_{2H}^{0} \varrho\vec{F} d\vec{r}
\end{equation}
\begin{equation}
p_0 - p_2H = \int_{2H}^{0} \varrho\omega^2 r dr + \int_{2H}^{0} \varrho\left(-g\right) dz
	\end{equation}

Fölbontjuk az integrálokat:

\begin{equation}
p_1 - p_2 = \varrho\omega^2 \left(\dfrac{\left(\dfrac{L}{2}\right)^2}{2} - \dfrac{\left(\dfrac{3L}{2}\right)^2}{2}\right) - \varrho g\left(0 - 2H\right)
\end{equation}

\begin{equation}
p_0 - p_2H = \varrho\omega^2 \left(\dfrac{L^2}{8} - \dfrac{9L^2}{8}\right) + \varrho g 2H
\end{equation}

\begin{equation}
p_0 - p_2H = -\varrho\omega^2 \dfrac{8L^2}{8} + \varrho g 2H
\end{equation}

Mivel tudjuk, hogy $p_1 = p_2$ ezért:

\begin{equation}
0 = -\varrho\omega^2 L^2 + \varrho g 2H
\end{equation}

\begin{equation}
\omega^2 L^2 = g 2H
\end{equation}

\begin{equation}
\omega = \dfrac{\sqrt{g 2H}}{L}
\end{equation}

\begin{equation}
\omega = \SI{7}{\dfrac{\radian}{\second}}
\end{equation}

Mivel már ismerjük az ${\omega}$ -t, ezért ebből már ki tudjuk számolni a fordulatszámot:

\begin{equation}
n = \dfrac{\omega}{2 \pi} = \SI{1,11}{\dfrac{1}{\second}} = \SI{66,88}{\dfrac{1}{\minute}}
\end{equation}

\subsubsection* {b) A nyomás meghatározása a $C$ pontban $\SI{}{\pascal}$ -ban:}

\begin{equation}
p_C - p_2H = \int_{2H}^{C} \varrho\vec{F} d\vec{r}
\end{equation}

\begin{equation}
p_C - p_2H = \int_{2H}^{C} \varrho\omega^2 r dr + \int_{2H}^{C} \varrho \left(-g\right) dz
\end{equation}

\begin{equation}
p_C - p_2H = \varrho\omega^2 \left(\dfrac{\left(\dfrac{3L}{2}\right)^2}{2} - \dfrac{\left(\dfrac{3L}{2}\right)^2}{2}\right) - \varrho g\left(0 - 2H\right)
\end{equation}

\begin{equation}
p_C - p_2H = \varrho g 2 H
\end{equation}

\begin{equation}
p_C = p_2H + \varrho g 2H
\end{equation}

\begin{equation}
p_C = \SI{107848}{\pascal}
\end{equation}


\newcommand{\aline}{\\\hline &&\rule{0cm}{1cm}}
	\newcounter{theyflines}
	
	Személyes adatok:
	\medskip
	
	\begin{tabular}{|p{3cm}|p{3cm}|p{3cm}|}
		\hline
		Név & Neptunkód & Szak
		\forloop{\aline}\\
		Sallér Anna & KPYXGU & Biomérnök BSc
		\forloop\\
		\hline
		
	\end{tabular}
	\medskip
	
