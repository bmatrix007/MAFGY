\section*{11. Határozza meg az ,,A" pontban a folyadék fajlagos energiatartalmát Nm/N-ban, illetve J/kg-ban, valamint a szükséges d átmérőt mm-ben.} 
$\dot{V}$=1,2m$^3$/s; p$_A$=6 bar; h=$z_A$=20m; D=0,6m; p$_0$=1 bar, g=9,81m/s$^2$, $\rho_V=10^3kg/m^3$

$$
d=\frac{2^{\frac{3}{4}}D\sqrt{Q}\rho_V^{\frac{1}{4}}}{(-D^4\pi^2p_0+D^4\pi^2p_A+D^4gh\pi^2\rho_V+8Q^2\rho_V)^{\frac{1}{3}}}=0,201702m
$$
\\
$$
v_A=\frac{4Q}{D^2\pi}=4,24413 ; \hspace*{10pt} v_B=\frac{4Q}{d^2\pi}=37,5555;
$$

$$
Y_A=\frac{v_A^2}{2}+\frac{p_A}{\rho_V}+z_Ag=805,2 J/kg
$$
\\
A fajlagos energiatartalom az A pontban: $e_A$=82,09Nm/N, illetve $Y_A$=805,2J/kg.