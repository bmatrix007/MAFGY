\section*{4/10. feladat: Szívócső számítása}


\addcontentsline{toc}{section}{4/10. feladat: Szívócső számítása}


\begin{tabular}{ | p{2cm} | p{14cm} | } 
	\hline
	Szerző & Bertók Dániel, AUDWOS \\ 
	\hline
	Szak & Biomérnök \\ 
	\hline
	Félév & 2019/2020 II. (tavaszi) félév \\ 
	\hline
\end{tabular}
\vspace{0.5cm}


\noindent Az ábrán látható szívócső teljes hossza

$l_\sum= \SI{10}{m}$,

$d= \SI{0,1}{m}$ átmérője,

$\lambda= 0,03$ a csősúrlódási tényező,

$c= \SI{3}{\meter\per\second}$ az áramlás sebessége,

Az idomdarabok veszteségtényezői:

lábszelep $\zeta_L= 3$, ívdarabok $\zeta_k= 0,5$.

$H= \SI{5}{m}$ magasság,

$\rho_v= \SI{1000}{\kilogram\per\meter\cubed}$,

$p_0= \SI{1}{bar}$,

$g= \SI{9,81}{\meter\per\second\squared}$ 

a) Mekkora a nyomás a szívócső A pontjában a szivattyú belépésénél?

b) Mekkora a szívócső egyenértékű csőhosszúsága?


% ide kell az ábra


\noindent\hrulefill