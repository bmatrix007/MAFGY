\section*{10. feladat: Szívócső számítása}


\addcontentsline{toc}{section}{10. feladat: Szívócső számítása}


\begin{tabular}{ | p{2cm} | p{14cm} | } 
	\hline
	Szerző & Bertók Dániel, AUDWOS \\ 
	\hline
	Szak & Biomérnök \\ 
	\hline
	Félév & 2019/2020 II. (tavaszi) félév \\ 
	\hline
\end{tabular}
\vspace{0.5cm}


\noindent Az ábrán látható szívócső teljes hossza

$l_\sum= \SI{11}{m}$,

$d= \SI{0,1}{m}$ átmérője,

$\lambda= 0,03$ a csősúrlódási tényező,

$c= \SI{3}{\meter\per\second}$ az áramlás sebessége,

Az idomdarabok veszteségtényezői:

lábszelep $\zeta_L= 3$, ívdarabok $\zeta_k= 0,5$.

$H= \SI{5}{m}$ magasság,

$\rho_v= \SI{1000}{\kilogram\per\meter\cubed}$,

$p_0= \SI{1}{bar}$,

$g= \SI{9,81}{\meter\per\second\squared}$ 

\noindent a) Mekkora a nyomás a szívócső \textbf{A} pontjában a szivattyú belépésénél?

\noindent b) Mekkora a szívócső egyenértékű csőhosszúsága?


% ide kell az ábra

\noindent\hrulefill

% ide kerül a megrajzolt ábra (hatóerők és ellenőrző felület)

\noindent Megoldás:

\noindent a)

\noindent Az áramvonal nem a lábszeleptől indul, hanem a nyugvó folyadék felszíntől, ezért $c_1= \SI{0}{\meter\per\second}$

\noindent A második  $\zeta_k$ valószínűleg felesleges, mert előtte kell a nyomást meghatározni.

\begin{equation}
p_0=\SI{1}{bar}=\SI{100000}{Pa}
\end{equation}

A veszteség tényezőt az alábbi egyenlet segítségével határozhatjuk meg:

\begin{equation}
Y_v=\sum_{i=1}^2\zeta_i{\frac{c^2}{2}}+\sum_{j=1}^1\lambda_j{\frac{l_j}{d_j}}{\frac{c^2}{2}}
\end{equation}

A csőkeresztmetszet felülete kiszámítható:

\begin{equation}
A={\frac{d^2\cdot{\pi}}{4}}
\end{equation}

Kezdeti és peremfeltételek:

\begin{equation}
\zeta_1=2\zeta_k,\zeta_2=\zeta_L, 
\end{equation}

\begin{equation}
\lambda_1=\lambda, l_1=\sum{l}
\end{equation}


Az általános egyenletbe behelyettesítve, az alábbi összefüggést kapjuk:

\begin{equation}
Y_v=(\zeta_L+2\zeta_k+\lambda\frac{l}{d})\frac{c^2}{2}
\end{equation}

\begin{equation}
Y_v=\frac{c^2\lambda\sum{l}}{2d}+c^2\zeta_k+\frac{c^2\zeta_L}{2}
\end{equation}

\begin{equation}
Y_v=\frac{{\SI{9}{\meter\squared\per\second\squared}}\cdot0,03\cdot\SI{11}{m}}{2\cdot\SI{0,1}{m}}+{\SI{9}{\meter\squared\per\second\squared}\cdot0,5+\frac{{\SI{9}{\meter\squared\per\second\squared}}\cdot3}{2}}
\end{equation}

\begin{equation}
\underline{\underline{Y_v=\SI{32,85}{\meter\squared\per\second\squared}}}}
\end{equation}

Veszteséges Bernoulli-egyenlet:
\begin{equation}
\frac{c_1^2}{2}+\frac{p_1}{\rho_v}+z_1g=\frac{c_2^2}{2}+\frac{p_2}{\rho_v}+z_2g+Y_v
\end{equation}

Kezdeti és peremfeltételek:

\begin{equation}
c_1=0, c_2=c,z_1=0, z_2=H, p_1=p_0
\end{equation}

Az egyenletetbe behelyettesítve:

\begin{equation}
\frac{p_0}{\rho_v}=\frac{c^2}{2}+gH+Y_v+\frac{p_2}{\rho_v}
\end{equation}

\begin{equation}
\frac{
	\SI{100 000}{Pa}
}{
\SI{1000}{\kilogram\per\meter\cubed}
}
=
\frac{{\SI{9}{\meter\squared\per\second\squared}}
}{
2
}
+
\SI{9,81}{\meter\per\second\squared}
\cdot
\SI{5}{m}
+
\SI{32,85}{\meter\squared\per\second\squared}
+
\frac{p_2}{\SI{1000}{\kilogram\per\meter\cubed}}
\end{equation}

Tehát az \textbf{A} pontban uralkodó nyomás:

\begin{equation}
\underline{\underline{p_2=\SI{13 600}{Pa}}}
\end{equation}

\noindent b)

Egyenértékű csőhosszúság számítása:

\begin{equation}
l_e=\sum{l}+(\zeta_L+2\zeta_k)\frac{d}{\lambda}
\end{equation}

\begin{equation}
l_e=\SI{11}{m}+(3+2\cdot0,5)\cdot\frac{\SI{0,1}{m}}{0,03}
\end{equation}

\begin{equation}
\underline{\underline{l_e=\SI{24,33}{m}}}
\end{equation}