\section*{3/9. feladat: Megtört lapra ható erő}
\addcontentsline{toc}{section}{3/9. feladat: Megtört lapra ható erő}
\begin{tabular}{ | p{2cm} | p{14cm} | } 
	\hline 
	Szerző & Péczer Fanni, RZMLZZ \\
	\hline
	Szak & Biomérnök Bsc. \\
	\hline	Félév & 2019/2020 II. (tavaszi) félév  \\
	\hline
\end{tabular}
\vspace{0.5cm}

\noindent Határozza meg az ábrán látható lapra ható eredő impulzuserőt N-ban, $ha$

$c=\SI{10}{\meter\per\second}$,

$u=\SI{2}{\meter\per\second}$,

$\rho_v=10^3 \frac{kg}{m^3}$,

$d=\SI{20}{mm}$,

$\alpha=90°!$

% ide kell az ábra


\noindent\hrulefill

% ide kerül a megrajzolt ábra (hatóerők és ellenőrző felület)

\noindent A vektor ábráról leolvasható, hogy a belépő $\vec{J}$ és a kilépő két $\vec{J}$ vektor összegének nagysága egyenlő!

\begin{equation}
J_{be} = 2 J_{ki}
\end{equation}

\begin{equation}
J_{ki} = \frac{J_{be}}{2}
\end{equation}

\noindent Ebben a feladatban a térfogatáramból $(q_v)$ indul ki a számításunk. A térfogatáram nem függ a cső keresztmetszetének nagyságától, csak a folyadék sebessége függ tőle. Ez alapján:

\begin{equation}
q_v=A\cdot{v}
\end{equation}

\noindent Az időegység alatt átáramlott tömeget nevezzük tömegáramnak. A tömegáram a térfogatáramból számítható, erre a feladat során még szükségünk lesz. Ez a következőképpen vezethető le:

\begin{equation}
q_m=q_v\cdot{\rho_v}
\end{equation}
\begin{equation}
q_m=A\cdot{v}\cdot{\rho_v}
\end{equation}
\begin{equation}
q_m={\rho_v}\cdot{\frac{d^2\cdot{\pi}}{4}}\cdot(c-u)
\end{equation}


\noindent A vektor ábrából és a tömegáram feleződéséből a következő egyenletet kapjuk:

\begin{equation}
J=J_{be}+2\cdot{J_{ki}\cdot{cos\left(\frac{\alpha}{2}\right)}}
\end{equation}

\noindent A (3.2)-es egyenletet behelyettesítve a (3.7)-be, az alábbi egyenletet kapjuk és ezt tovább alakítjuk:

\begin{equation}
J=J_{be}+2\cdot\frac{J_{be}}{2}\cdot{cos\left(\frac{\alpha}{2}\right)}
\end{equation}

\begin{equation}
\frac{J}{2}=\frac{J_{be}}{2}+\frac{J_{be}}{2}\cdot{cos\left(\frac{\alpha}{2}\right)}
\end{equation}

\begin{equation}
J=J_{be}+{cos\left(\frac{\alpha}{2}\right)}\cdot{J_{be}}
\end{equation}

\noindent Ahhoz, hogy a (3.10)-es egyenlettel eredményt számolhassunk, ki kell számolni a belépő impulzus értéket. Ehhez szükség van a tömegáram $(q_m)$ és, az áramlási sebesség $(v)$ ismeretére. Ezek a számítások a következőek:

\begin{equation}
J_{be}=q_m\cdot{v}
\end{equation}

\begin{equation}
q_m=\rho_v\cdot{\frac{d^2\cdot{\pi}}{4}}\cdot(c-u)=1000\frac{kg}{m^3}\cdot{\frac{0,02^2{m}\cdot{\pi}}{4}}\cdot(10\frac{m}{s}-2\frac{m}{s})=2,5133  {\frac{kg\cdot{m}}{s^2}}
\end{equation}

\begin{equation}
v=c-u=10\frac{m}{s}-2\frac{m}{s}=8\frac{m}{s}
\end{equation}

\noindent Tehát a (3.12) és (3.13) egyenletek eredményeit behelyettesítjük a (3.11)-es egyenletbe, így megkapjuk a belépő impulzus mennyiséget:

\begin{equation}
J_{be}=q_m\cdot{v}=2,5133\frac{kg\cdot{m}}{s^2}\cdot{8}\frac{m}{s}=20,1064\frac{kg\cdot{m}}{s^2}
\end{equation}

\noindent A végeredmény a megadott $\alpha$ és a (3.14)-es egyenlet eredményének segítségével számítható ki:

\begin{equation}
J=J_{be}+{cos\left(\frac{\alpha}{2}\right)}\cdot{J_{be}}=20,1064\frac{kg\cdot{m}}{s^2}+{cos\left(\frac{90°}{2}\right)}\cdot{20,1064\frac{kg\cdot{m}}{s^2}}=34,3237 \frac{kg\cdot{m}}{s^2}
\end{equation}


\noindent\centering {Tehát az ábrán látható megtört lapra ható eredő impulzuserő mennyisége $\underline{\underline{34,3237 N}}$}.

\pagebreak
