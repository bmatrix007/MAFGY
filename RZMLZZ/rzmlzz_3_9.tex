
\section*{3/9. feladat: Megtört lapra ható erő}
\addcontentsline{toc}{section}{3/9. feladat: Megtört lapra ható erő}
\begin{tabular}{ | p{2cm} | p{14cm} | } 
	\hline 
	Szerző & Péczer Fanni, RZMLZZ \\
	\hline
	Szak & Biomérnök Bsc. \\
	\hline	Félév & 2019/2020 II. (tavaszi) félév  \\
	\hline
\end{tabular}
\vspace{0.5cm}

\noindent Határozza meg, a bal oldali ábrán látható,
lapra ható eredő impulzuserőt N-ban, $ha$
az egyik sebesség $c=\SI{10}{\meter\per\second}$, míg a másik sebesség $u=\SI{2}{\meter\per\second}$	a víz sűrűsége $\rho_v=\SI{e3}{\kilo\gram\per\meter\cubed}=\SI{1000}{\kilo\gram\per\meter\cubed}$	a cső átmérője $d=\SI{20}{\milli\meter}=\SI{0,02}{\meter}$ és a törés szöge	$\alpha=\SI{90}{\degree}$!

\begin{figure}[ht]
\begin{subfigure}[b]{0.5\textwidth}
	\centering
	\label{figure:sm}
		\begin{tikzpicture}
			\draw[step=1cm, white, very thin] (-5, -1) grid (5, 6);
			\draw[black, ultra thick] (0, 0) -- (2.5, 2.5);
			\draw[black, ultra thick] (2.5, 2.5) -- (0, 5);
			\draw[black, ultra thick] (-3.25, 2) rectangle (-1.25, 3);
			\draw[color=black, semithick, dashed] (-1.25,2) -- (1.25, 2);
			\draw[color=black,semithick,  dashed] (-1.25,2.25) -- (1.75, 2.25);
			\draw[color=black,semithick,  dashed] (-1.25,2.5) -- (2.25, 2.5);
			\draw[color=black,semithick,  dashed] (-1.25,2.75) -- (1.75, 2.75);
			\draw[color=black,semithick,  dashed] (-1.25,3) -- (1.25, 3);
			\draw[color=black, semithick, dashed, angle=45] (1.25, 2) -- (0,0.75);
			\draw[color=black, semithick, dashed, angle=45] (1.75, 2.25) -- (0,0.5);
			\draw[color=black, semithick, dashed, angle=45] (2.25, 2.5) -- (0,0.25);
			\draw[color=black, semithick, dashed, angle=135] (1.25, 3) -- (0,4.25);
			\draw[color=black, semithick, dashed, angle=135] (1.75, 2.75) -- (0,4.5);
			\draw[color=black, semithick, dashed, angle=135] (2.25, 2.5) -- (0,4.75);
			\draw (2.5, 3.5) arc [radius=1.0, start angle=90, end angle=270];
			\draw[thin, dash dot] (-4, 2.5) -- (3.5, 2.5);
			\node[west] at (2,2.5) {$\alpha$};
			\node[south] at (-0.25,2.5) {C};
			\node[south] at (3,2.75) {U};
			\draw[->, black] (-1.25, 2.5) -- (-0.25, 2.5);
			\draw[->, black] (2.75, 2.5) -- (3.5, 2.5);
			\draw[->, black] (-3, 1) -- (-3, 2);
			\draw[black, semithick] (-3, 2) -- (-3, 3);
			\draw[->, black] (-3, 4) -- (-3, 3);
			\draw[black, semithick] (-3, 3.5) -- (-2, 3.5) 
			node[black, anchor=south east]{$\phi{d}$};
			\draw[black, semithick] (-1, 3.5) -- (0, 3.5);
			\draw[black, semithick] (0, 3.5) -- (0.5, 2.5);
			\draw[black, semithick] (-0.25, 3.5) -- (-0.25, 3.5) node[black, anchor=south east]{$\rho_V$};
			\draw [->, black] (2.5, 3.5) arc [radius=1.0, start angle=90, end angle=135];
			\draw [->, black] (2.5, 1.5) arc [radius=1.0, start angle=270, end angle=225];
		\end{tikzpicture}
		\caption{Megtört lapra ható erő}
\end{subfigure}
\begin{subfigure}[b]{0.5\textwidth}
\centering
\begin{tikzpicture}
	\draw[step=1cm, white, very thin] (-5, -1) grid (5, 6);
	\draw[black, ultra thick] (0, 0) -- (2.5, 2.5);
	\draw[black, ultra thick] (2.5, 2.5) -- (0, 5);
	\draw[black, ultra thick] (-3.25, 2) rectangle (-1.25, 3);
	\draw[color=black, semithick, dashed] (-1.25,2) -- (1.25, 2);
	\draw[color=black,semithick,  dashed] (-1.25,2.25) -- (1.75, 2.25);
	\draw[color=black,semithick,  dashed] (-1.25,2.5) -- (2.25, 2.5);
	\draw[color=black,semithick,  dashed] (-1.25,2.75) -- (1.75, 2.75);
	\draw[color=black,semithick,  dashed] (-1.25,3) -- (1.25, 3);
	\draw[color=black, semithick, dashed, angle=45] (1.25, 2) -- (0,0.75);
	\draw[color=black, semithick, dashed, angle=45] (1.75, 2.25) -- (0,0.5);
	\draw[color=black, semithick, dashed, angle=45] (2.25, 2.5) -- (0,0.25);
	\draw[color=black, semithick, dashed, angle=135] (1.25, 3) -- (0,4.25);
	\draw[color=black, semithick, dashed, angle=135] (1.75, 2.75) -- (0,4.5);
	\draw[color=black, semithick, dashed, angle=135] 	(2.25, 2.5) -- (0,4.75);
	\draw (2.5, 3.5) arc [radius=1.0, start angle=90, end angle=270];
	\draw[thin, dash dot] (-4, 2.5) -- (3.5, 2.5);
	\node[west] at (2,2.5) {$\alpha$};
	\node[south] at (-0.25,2.5) {C};
	\node[south] at (3,2.75) {U};
	\draw[->, black] (-1.25, 2.5) -- (-0.25, 2.5);
	\draw[->, black] (2.75, 2.5) -- (3.5, 2.5);
	\draw[->, black] (-3, 1) -- (-3, 2);
	\draw[black, semithick] (-3, 2) -- (-3, 3);
	\draw[->, black] (-3, 4) -- (-3, 3);
	\draw[black, semithick] (-3, 3.5) -- (-2, 3.5) 
	node[black, anchor=south east]{$\phi{d}$};
	\draw[black, semithick] (-1, 3.5) -- (0, 3.5);
	\draw[black, semithick] (0, 3.5) -- (0.5, 2.5);
	\draw[black, semithick] (-0.25, 3.5) -- (-0.25, 3.5) node[black, anchor=south east]{$\rho_V$};
	\draw [->, black] (2.5, 3.5) arc [radius=1.0, start angle=90, end angle=135];
	\draw [->, black] (2.5, 1.5) arc [radius=1.0, start angle=270, end angle=225];
	\draw[red, ultra thick, dashed](0, 0) -- (0, 0.75) --(1.25, 2) -- (0.75, 2) -- (0.75, 3) -- (1.25, 3) -- (0, 4.25) -- (0, 5) -- (2.5, 2.5) -- (0, 0);
	\draw[->, blue, ultra thick] (0.75, 2.5) -- (-1, 2.5) 
	node[midway, anchor=south west, xshift={-6mm}, yshift={1mm}]{$\vec{J}_{BE}$};
	\draw[->, blue, ultra thick] (0, 4.5) -- (-1, 5.5) 
	node[midway, anchor=south west, xshift={-3mm}, yshift={1mm}]{$\vec{J}_{KI}$};
	\draw[->, blue, ultra thick] (0, 0.5) -- (-1, -0.5) 
	node[midway, anchor=south west, xshift={-6mm}, yshift={1mm}]{$\vec{J}_{KI}$};
	\end{tikzpicture}
	\caption{Vektor ábra}
\end{subfigure}
\caption{A feladatban megadott ábra és az elkészített vektor ábra}
\label{figure:fre}
\end{figure}


\noindent\hrulefill

\noindent Első lépésben jelölni kell az ábrára az ellenőrző felületet, ezt pirossal jelölöm, és a hatóerőket, ezek pedig kék színnel jelennek meg. A megrajzolt vektor ábra a vonal felett jobb oldalon látható (b).

\noindent A vektor ábráról leolvasható, hogy a belépő $\vec{J_{BE}}$ és a kilépő két $\vec{J_{KI}}$ vektor összegének nagysága egyenlő!

\begin{equation}
J_{be} = 2 J_{ki}
\end{equation}

\begin{equation}
J_{ki} = \frac{J_{be}}{2}
\end{equation}

\noindent Ebben a feladatban a térfogatáramból $(q_v)$ indul ki a számításunk. A térfogatáram nem függ a cső keresztmetszetének nagyságától, csak a folyadék sebessége függ tőle. Ez alapján:

\begin{equation}
q_v=A\cdot{v}
\end{equation}

\noindent Az időegység alatt átáramlott tömeget nevezzük tömegáramnak. A tömegáram a térfogatáramból számítható, erre a feladat során még szükségünk lesz. Ez a következőképpen vezethető le:

\begin{equation}
q_m=q_v\cdot{\rho_v}
\end{equation}
\begin{equation}
q_m=A\cdot{v}\cdot{\rho_v}
\end{equation}
\begin{equation}
q_m={\rho_v}\cdot{\frac{d^2\cdot{\pi}}{4}}\cdot(c-u)
\end{equation}


\noindent A vektor ábrából és a tömegáram feleződéséből a következő egyenletet kapjuk:

\begin{equation}
J=J_{be}+2\cdot{J_{ki}\cdot{cos\left(\frac{\alpha}{2}\right)}}
\end{equation}

\noindent A (3.2)-es egyenletet behelyettesítve a (3.7)-be, az alábbi egyenletet kapjuk és ezt tovább alakítjuk:

\begin{equation}
J=J_{be}+2\cdot\frac{J_{be}}{2}\cdot{cos\left(\frac{\alpha}{2}\right)}
\end{equation}

\begin{equation}
\frac{J}{2}=\frac{J_{be}}{2}+\frac{J_{be}}{2}\cdot{cos\left(\frac{\alpha}{2}\right)}
\end{equation}

\begin{equation}
J=J_{be}+{cos\left(\frac{\alpha}{2}\right)}\cdot{J_{be}}
\end{equation}

\noindent Ahhoz, hogy a (3.10)-es egyenlettel eredményt számolhassunk, ki kell számolni a belépő impulzus értéket. Ehhez szükség van a tömegáram $(q_m)$ és, az áramlási sebesség $(v)$ ismeretére. Ezek a számítások a következőek:

\begin{equation}
J_{be}=q_m\cdot{v}
\end{equation}

\begin{equation}
q_m=\rho_v\cdot{\frac{d^2{\pi}}{4}}\cdot(c-u)=\SI{1000}{\kilo\gram\per\meter\cubed}\cdot{\frac{{\SI{4e-4}{\meter}\cdot{\pi}}}{4}}\cdot(\SI{10}{\meter\per\second}-\SI{2}{\meter\per\second})=\SI{2,5133}{\kilo\gram\per\second}
\end{equation}

\begin{equation}
v=c-u=\SI{10}{\meter\per\second}-\SI{2}{\meter\per\second}=\SI{8}{\meter\per\second}
\end{equation}

\noindent Tehát a (3.12) és (3.13) egyenletek eredményeit behelyettesítjük a (3.11)-es egyenletbe, így megkapjuk a belépő impulzus mennyiséget:

\begin{equation}
J_{be}=q_m\cdot{v}=\SI{2,5133}{\kilo\gram\per\second}\cdot{\SI{8}{\meter\per\second}}=\SI{20,1064}{\kilo\gram\meter\per\second\squared}
\end{equation}

\noindent A végeredmény a megadott $\alpha$ és a (3.14)-es egyenlet eredményének segítségével számítható ki:

\begin{equation}
J=J_{be}+{cos\left(\frac{\alpha}{2}\right)}\cdot{J_{be}}=\SI{20,1064}{\kilo\gram\meter\per\second\squared}+{cos\left(\frac{\SI{90}{\degree}}{2}\right)}\cdot{\SI{20,1064}{\kilo\gram\meter\per\second\squared}=\SI{34,3237}{\kilo\gram\meter\per\second\squared}}
\end{equation}


\noindent\centering {Tehát az ábrán látható megtört lapra ható eredő impulzuserő mennyisége $\underline{\underline{\SI{34,3237}{\newton}}}$}.

\pagebreak
