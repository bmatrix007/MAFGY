\section*{1/21. feladat: Elzáró szerkezet}
\addcontentsline{toc}{section}{1/21. feladat}
\begin{tabular}{ | p{2cm} | p{14cm} | } 
	\hline
	Szerző & Talpai Szindarella, R41KZ8 \\ 
	\hline
	Szak & Vegyészmérnök \\ 
	\hline
	Félév & 2019/2020 II. (tavaszi) félév \\ 
	\hline
\end{tabular}
\vspace{0.5cm}

\noindent  Az ábrán látható egy $ A $ felületű lemez zár le, amelynek nyitása a $ G $ súlynak egy vízszintes karon való mozgatásával szabályozható. Mekkora $ x $ távolsággal kell a súlyt elmozdítani, hogy a folyadék éppen ne folyjon ki, ha a folyadékszint $\Delta h$ magassággal nő?

\begin{flushleft}
	$\varrho_{L}$ $\cong$ \SI{0}{\kilogram\per\meter\cubed},\\
	h = \SI{60}{\centi\meter},\\
	$\Delta$h = \SI{5}{\centi\meter},\\
	a = \SI{20}{\centi\meter}.
\end{flushleft}

\noindent\hrulefill

\noindent A  súly által kifejtett forgatónyomaték és a hidrosztatikai nyomás által kifejtett erők egyenlőek.

\begin{equation}
	p = \dfrac{F}{A}
	\quad 
	\Rightarrow
	\quad 
	F = pA
\end{equation}

\noindent A sűrűség kifejezve paraméteresen:
\begin{equation}
	 \varrho ghA = aG
	 \quad 
	 \Rightarrow
	 \quad 
	 \varrho = \dfrac{G}{A} \SI{0,03398}{\meter\per\second\squared}
\end{equation}

\noindent Az elmozdított esetre felírt erőkből kitudjuk számolni az $x$ értékét:
\begin{equation}
	\dfrac{\bcancel{G}}{\bcancel{A}}
	 \SI{0,03398}{\meter\per\second\squared} g \left( h + \Delta h\right) \bcancel{A}
	 = 
	 \left( a+x\right) \bcancel{G}
\end{equation}

\begin{equation}
	\SI{21,67}{\centi\meter} = \SI{20}{\centi\meter} + x 
\end{equation}

\begin{equation}
	x = \SI{1,67}{\centi\meter}
\end{equation}

