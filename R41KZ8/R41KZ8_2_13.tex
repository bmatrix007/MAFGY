\section*{2/13. feladat: Forgó könyökcső vízszállítása}
\addcontentsline{toc}{section}{2/13. feladat}
\begin{tabular}{ | p{2cm} | p{14cm} | } 
	\hline
	Szerző & Talpai Szindarella, R41KZ8 \\ 
	\hline
	Szak & Vegyészmérnök \\ 
	\hline
	Félév & 2019/2020 II. (tavaszi) félév \\ 
	\hline
\end{tabular}
\vspace{0.5cm}

\noindent Határozza meg a vízzel feltöltött könyökcső vízszállítását $\si{\liter\per\second}$-ban, ha
\begin{flushleft}
	n = \SI{400}{1\per\minute},\\
	d = \SI{30}{\milli\meter} = \SI{0,03}{\meter},\\
	h = \SI{0,15}{\meter},\\
	r = \SI{0,2}{\meter},\\
	$\varrho_{v}$ = \SI{e3}{\kilogram\per\meter\cubed},\\
	g = \SI{9,81}{\meter\per\second\squared},\\
	$p_{0}$ = \SI{1}{\bar} = \SI{e5}{\pascal}.
\end{flushleft}
\vspace{2mm}

\noindent Az áramlás veszteségmentesnek tekinthető.

\subsubsection*{ Megoldási útmutatás:}

\vspace{2mm}

\begin{flushleft}
	Jelölje ki a vonatkoztatási szintet és a vizsgálandó pontokat!
	Írja fel a térerő változását integrál alakban és végezze el az integrálást egy célszerűen választott koordináták mentén!
\end{flushleft}

\noindent\hrulefill

\subsubsection{Függőleges tengely körül forgó rendszer}
A függőleges tengely körül forgó vonatkoztatási rendszer miatt henger-koordinátarendszert alkalmazunk. A térerő: 
\begin{equation}
	\vec{f} = \vec{g} - \vec{a}_{cp}
	=
	\mtrx{\omega ^{2} r \\ 0 \\ -g} \mtrx{\dif r \\ rd \vartheta \\ \dif z}
\end{equation}

\subsubsection{A nyomáskülönbség számítása}
\noindent A szögsebesség meghatározása:
\begin{equation}
	n = \SI{400}{1\per\minute}
	\quad 
	\Rightarrow
	\quad 
	\omega = \dfrac{2 \pi}{60} 400 = \SI{41,893}{\radian\per\second}
\end{equation}

\noindent A két pont közti nyomáskülönbség a hidrostatika alapegyenletének integrális alakjából kiindulva: 
\begin{equation}
 	p_{A} - p_{B} = \int_{B}^{A} \varrho_{v} \vec{f} \cdot \dif \vec{r}
 	=
 	\varrho_{v} \int_{r}^{0} \omega ^{2} r \dif r 
 	+ 
 	\varrho_{v} \int_{h}^{0} \left( -g \right) \dif z 	
\end{equation}

\begin{equation}
	p_{A} - p_{B} = \varrho_{v} \omega ^{2} \left[ \dfrac{r^{2}}{2}\right]_{r}^{0}
	+
	\varrho_{v} g \left[ z\right]_{0}^{h}
	=
	\varrho_{v} \omega ^{2} \left( - \dfrac{r^{2}}{2}\right)
	+ \varrho_{v} g h
\end{equation}

\begin{equation}
	p_{A} = p_{B} - \varrho_{v} \omega ^{2} \dfrac{r^{2}}{2}
	+
	\varrho_{v} g h	
	=
	\SI{66 376}{\pascal}
\end{equation}
\subsubsection{Bernoulli-tétel alapján}
\noindent Az áramlás veszteségmentesnek tekinthető, ezért ennek megfelelő Bernoulli-egyenletet alkalmazzuk:
\begin{equation}
	\dfrac{v_{1}^{2}}{2} + \dfrac{p_{1}}{\varrho} + \dfrac{1}{2} \omega ^{2} r_{1}^{2} + gz_{1}
	=
	\dfrac{v_{2}^{2}}{2} + \dfrac{p_{2}}{\varrho} + \dfrac{1}{2} \omega ^{2} r_{2}^{2} + gz_{2}
\end{equation}

\noindent Kezdeti és peremfeltételek: 
\begin{equation*}
	v_{1} = \SI{0}{\meter\per\second}  
	\quad
	p_{1} = p_{0} = \SI{e5}{\pascal}  
	\quad
	\omega = \SI{41,893}{\radian\per\second}
	\quad
	r_{1} = \SI{0,2}{\meter}
	\quad
	z_{1} = h
\end{equation*}

\begin{equation*}
	v_{2} = ? \si{\meter\per\second} 
	\quad
	p_{2} = \SI{66 376}{\pascal}  
	\quad
	\omega = \SI{41,893}{\radian\per\second}
	\quad
	r_{2} = \SI{0}{\meter}
	\quad
	z_{2} = \SI{0}{\meter}
\end{equation*}

\noindent Kifejezve a $v_{2}$-t : 

\begin{equation}
	v_{2} = \sqrt{\left( \dfrac{p_{1}-p_{2}}{\varrho} + \dfrac{1}{2} \omega ^{2} r_{1}^{2} + g h \right) 2 }
\end{equation}

\noindent Az egyenletbe behelyettesítve:

\begin{equation}
	v_{2} = \sqrt{\left( \dfrac{\SI{e5}{\pascal}-\SI{66 376}{\pascal}}{\SI{e3}{\kilogram\per\meter\cubed}}
	+ 
	\dfrac{1}{2} {41,893}^{2} \si{\radian\per\second} {0,2}^{2} \si{\meter}
	+ 
	\SI{9,81}{\meter\per\second\squared} \SI{0,15}{\meter} \right) 2 }
	= 
	\SI{11,85}{\meter\per\second}
\end{equation}

\subsubsection{A térfogatáram}
\noindent A folyadék szállításának pillanatnyi gyorsaságát adja meg az időegység alatt átáramlott térfogat alakjával egy adott keresztmetszeten.\\
\noindent A keresztmetszet kiszámítható:
\begin{equation}
	A = \dfrac{d^{2} \pi}{4}
\end{equation}

\noindent A könyökcső vízszállítása:
\begin{equation}
	\dot{V} = v_{2} A =  \SI{11,85}{\meter\per\second} \quad 7,07 \cdot 10^{-4} \si{\meter\squared}
	= 
	8,38 \cdot 10^{-3} \si{\meter\cubed\per\second}
\end{equation}

\noindent Átszámítva $\si{\liter\per\second}$-ban:
\begin{equation}
	\dot{V} = \SI{8,38}{\liter\per\second}
\end{equation}
