% !TeX TS-program = xelatex

\documentclass[11pt, a4paper]{report}

% Dokumentum adatok
% =================
\author{}
\title{Műszaki áramlástan feladatgyűjtemény}

% A közös fájlok beszúrása
% ========================
\newcommand*{\JakiFolder}{./JAKI}%

% A közös fájlok a JAKI tárolóban vannak, amit az MHFGY-vel 
% közös mappába kell letölteni (clone/pull).
\input{\JakiFolder/JakiAlap.tex}				% Formázás, csomagok
\input{\JakiFolder/JakiTikz.tex}				% Rajzoló parancsok

% A dokumentum kezdete
% ====================
\begin{document}

% Címoldal
\begin{titlepage}
	\centering
	\begin{hyphenrules}{nohyphenation}
		{\scshape\LARGE Pannon Egyetem \par}
		{\scshape\LARGE Mérnöki Kar \par}
		\vspace{1cm}
		{\scshape\Large Segédlet\par}
		\vspace{1.5cm}
		\parbox{8cm}{{\centering\huge\bfseries{\JakiTitle} \par}}
		\vspace{2cm}
		{\Large\itshape\JakiAuthor\par}
		\vfill
		Műszaki áramlástan \par
		Műszaki áramlástan és hőtan I.\par
		Műszaki áramlás- és hőtan

		\vfill

		% A lap alja
		{\large \today\par}
	\end{hyphenrules}
\end{titlepage}

% Tartalomjegyzék
% A frissüléséhez általában kétszer kell lefordítani a dokumentumot
\tableofcontents


% Bevezetés
\chapter*{Alapadatok}
\addcontentsline{toc}{chapter}{Alapadatok}		% Számozatlan címsor és tartalomjegyzék-bejegyzés

\section*{A tárgy adatai}
\addcontentsline{toc}{section}{A tárgy adatai}

\begin{tabular}{ l l }
Név: & Műszaki áramlástan \\
Kód: & VEMKGEB143H \\
Kreditérték: & 3 (2 elmélet, 1 gyakorlat) \\
Követelmény típus: & vizsga \\
Szervezeti egység: & Gépészmérnöki Intézet \\
Előadás látogatása: & kötelező \\
Gyakorlat látogatása: & kötelező \\
Számonkérés: & a félév végén zárthelyi, írásbeli és szóbeli vizsga \\
\end{tabular}

\section*{A segédlet célja}
\addcontentsline{toc}{section}{A segédlet célja}

A segédlet célja.

A segédlet kidolgozása még folyamatban van.


\section*{Ajánlott szakirodalom}
\addcontentsline{toc}{section}{Ajánlott szakirodalom}

\begin{itemize}
	\item Irodalom.
\end{itemize}


% 1. fejezet
% ==========
\part{title}\chapter{Hidrostatika}

% 1/1

{\tiny {\tiny }}
% 1/2


% 1/3


% 1/4


% 1/5


% 1/6


% 1/7


% 1/8


% 1/9


% 1/10


% 1/11


% 1/12


% 1/13


% 1/14


% 1/15


% 1/16


% 1/17


% 1/18


% 1/19


% 1/20


% 1/21
\section*{1/21. feladat: Elzáró szerkezet}
\addcontentsline{toc}{section}{1/21. feladat}
\begin{tabular}{ | p{2cm} | p{14cm} | } 
	\hline
	Szerző & Talpai Szindarella, R41KZ8 \\ 
	\hline
	Szak & Vegyészmérnök \\ 
	\hline
	Félév & 2019/2020 II. (tavaszi) félév \\ 
	\hline
\end{tabular}
\vspace{0.5cm}

\noindent  Az ábrán látható egy $ A $ felületű lemez zár le, amelynek nyitása a $ G $ súlynak egy vízszintes karon való mozgatásával szabályozható. Mekkora $ x $ távolsággal kell a súlyt elmozdítani, hogy a folyadék éppen ne folyjon ki, ha a folyadékszint $\Delta h$ magassággal nő?

\begin{flushleft}
	$\varrho_{L}$ $\cong$ \SI{0}{\kilogram\per\meter\cubed},\\
	h = \SI{60}{\centi\meter},\\
	$\Delta$h = \SI{5}{\centi\meter},\\
	a = \SI{20}{\centi\meter}.
\end{flushleft}

\noindent\hrulefill

\noindent A  súly által kifejtett forgatónyomaték és a hidrosztatikai nyomás által kifejtett erők egyenlőek.

\begin{equation}
	p = \dfrac{F}{A}
	\quad 
	\Rightarrow
	\quad 
	F = pA
\end{equation}

\noindent A sűrűség kifejezve paraméteresen:
\begin{equation}
	 \varrho ghA = aG
	 \quad 
	 \Rightarrow
	 \quad 
	 \varrho = \dfrac{G}{A} \SI{0,03398}{\meter\per\second\squared}
\end{equation}

\noindent Az elmozdított esetre felírt erőkből kitudjuk számolni az $x$ értékét:
\begin{equation}
	\dfrac{\bcancel{G}}{\bcancel{A}}
	 \SI{0,03398}{\meter\per\second\squared} g \left( h + \Delta h\right) \bcancel{A}
	 = 
	 \left( a+x\right) \bcancel{G}
\end{equation}

\begin{equation}
	\SI{21,67}{\centi\meter} = \SI{20}{\centi\meter} + x 
\end{equation}

\begin{equation}
	x = \SI{1,67}{\centi\meter}
\end{equation}





\chapter{Veszteségmentes csőáramlások}

% 2/1 Szódásszifon


% 2/2 Függőleges Venturi-mérő


% 2/3 Vízszintes Venturi-mérő


% 2/4 Túlnyomásos tartályrendszer


% 2/5 Vízrakéta


% 2/6 Szivornya


% 2/7 Tartály oldalán kiömlő vízsugár


% 2/8 Pipacső vízszállítása


% 2/9 Vízszintemelkedés pipacsőben


% 2/10 Csővezetékben áramló levegő


% 2/11 Áramló folyadék tömegfajlagos energiája


% 2/12 Dugattyú megengedhető legnagyobb gyorsulása


% 2/13 Forgó könyökcső vízszállítása
\section*{2/13. feladat: Forgó könyökcső vízszállítása}
\addcontentsline{toc}{section}{2/13. feladat}
Határozza meg a vízzel feltöltött könyökcső vízszállítását $\ell$/s-ban, ha

n = \SI{400}{1/\minute},

d = \SI{30}{\mm},

r = \SI{0,2}{\m},

$\varrho_{v} = \SI{e3}{\kilogram/\meter\cubed} $,

g = \SI{9,81}{\meter/\second\squared},

p_{0} = \SI{1}{\bar}.

\vspace{2mm}

\noindent Az áramlás veszteségmentesnek tekinthető.

\subsubsection*{ Megoldási útmutatás:}

\vspace{2mm}

\begin{flushleft}
	Jelölje ki a vonatkoztatási szintet és a vizsgálandó pontokat!
	Írja fel a térerő változását integrál alakban és végezze el az integrálást egy célszerűen választott koordináták mentén!
\end{flushleft}

\noindent\hrulefill




\chapter{Folyadékáramlás erőhatásai, kifolyás tartályból}

% 3/1 Vízsugár erőhatása


% 3/2 Ívelt lapátra érkező vízsugár


% 3/3 Éket emelő vízsugár


% 3/4 Rugót tehermentesítő vízsugár


% 3/5 Átáramlott ívcsőre ható erő


% 3/6 Bővülő ívcsőre ható erő


% 3/7 Átáramlott szűkítőre ható erő


% 3/8 Coandă-hatás


% 3/9 Megtört lapra ható erő


% 3/10 Borda-féle kifolyónyílással ellátott tartály


% 3/11 Hengeres tartály kiürülési ideje


% 3/12 Kifolyás kúpos tartályból


% 3/13 Kifolyó olaj nyúlóssága


% 3/14 Tartályok közötti vízszint-kiegyenlítődés


\chapter{Valós folyadék áramlása csővezetékben}

% 4/1 Síklap vontatás folyadékrétegen


% 4/2 Nyomásesés vizet szállító csővezetéken


% 4/3 Veszteséges áramlás részben kitöltött csatornában


% 4/4 Veszteséges áramlás gyűrű keresztmetszetű csatornában


% 4/5 Csővezetékrendszer


% 4/6 Keveredő áramlás számítása


% 4/7 Csővezetékrendszer szivattyúval


% 4/8 Pelton-turbina nyomócsövének vesztesége


% 4/9 Nyomás meghatározás csővezetékrendszerben


% 4/10 Szívócső számítása


% 4/11 Csővezeték méretezése


\chapter{Összenyomhatatlan folyadék egyméretű áramlása}

% 5/1 Pitot--Prandtl-cső


% 5/2 Billenőgyűrűs manométer


% 5/3 Venturi-mérő veszteséges számítása


% 5/4


% 5/5


% 5/6


% 5/7



\end{document}